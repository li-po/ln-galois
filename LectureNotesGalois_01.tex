\chapter{Generalities on algebraic extensions}

Ekaterina Amerik is working at the mathematical department of Higher School of Economics.

This is a course about field extensions and we assume familiar the basic notions of abstract algebra, like groups, rings, fields, modules, ideals, and their basic properties. We also assume a certain knowledge of linear algebra. All rings we consider will be commutative, associative, and with unity. 

\section{Field extensions: examples}

\begin{definition}
$K$, $L$ fields. $L$ is said to be an \textbf{extension} of $K$, if $K$ subfield of $L$.
\end{definition}
An equivalent definition is:
\begin{definition}
$L$ is an \textbf{extension} of $K$ if $L$ is a $K$-algebra. 
\end{definition}

A $K$-algebra $A$ is a ring with a structure of a $K$-module such that these two structures are compatible: the multiplication $A\times A \to A$ is $K$-bilinear, $(k_1 a_1) (k_2 a_2) = k_1 k_2  a_1 a_2$, where $k_i \in K$, $a_i \in A$. 

Why is a field containing K as a subfield the same thing as a K-algebra?

Defining a K-algebra structure on A is the same as defining a homomorphism of rings $f: K \to A$. Given a $K$-algebra, we can define a homomorphism $f(k) = k \cdot 1$. And conversely, if I have a homomorphism of rings $f: K \to A$, I can define a K-algebra structure by setting $ka = f(k) \cdot a$. 

If $A$ is a field $L$, then any homomorphism $f:K \to L$ is injective. There are several ways to see this. 1) $f(k)$ is always invertible: $1 = f(1) = f(kk^{-1}) = f(k) f(k^{-1})$. So $f(k)\neq 0$ if $k \neq 0$. In particular $f$ is injective. 2) $\text{Ker }f$ is always an ideal: a field does not have nontrivial ideals, the only ideals are $(0)$ and $(1)=K$. 

If you don't know this, you are strongly encouraged to do it as an exercise. 

\begin{example}
$\mathbb{C}$ is an extension of $\mathbb{R}$. And $\mathbb{R}$ is an extension of $\mathbb{Q}$. 
\end{example}

\begin{example}
Any field has what is called a characteristic, so if $L$ is a field, there are two possibilities: 

1) If you take the unit element and start adding it to itself, you never obtain $1+1+\cdots +1 \neq 0$. In this case, we say that the $\text{char }K = 0$. And we see that in this case of course, $\mathbb{Z} \subset L \implies \mathbb{Q} \subset L$. So L is an extension of $\mathbb{Q}$.

2) Or, if you take the unit element and start adding it to itself, you obtain 0 at certain point $\sum_{m\text{ times}} 1 = 0$. The minimal $m$ with this property is prime. $m$ is the \textbf{characteristic} of this field. Then $L$ does not contain $\mathbb{Z}$, it contains $\mathbb{Z}/p\mathbb{Z}$. For $p$ prime $\mathbb{Z}/p\mathbb{Z}$ is a field. To emphasize its field structure we denote it by $\mathbb{F}_p$. $L$ is an extension of $\mathbb{F}_p$.

One calls $\mathbb{Q}$ and $\mathbb{F}_p$ the prime fields. They don't contain any proper subfields.

Any field is an extension of one prime field.
\end{example}

This example is very important:
\begin{example}
So let's take the ring of polynomials in one variable over $K$. And let's consider the quotient ring by an ideal generated by an irreducible polynomial $P$: $K[x]/(P)$. Then $K[x]/(P)$ is a field. 

There are two ways to see these. 

\textbf{Way 1.} If $Q$ is a polynomial which is not a multiple of $P$ ($Q\notin (P)$), then $Q$ is prime to $P$ ($gcd(Q,P)=1$).  And then you have the Bézout identity. $\exists A,B \in K[x]$, such that $AP + BQ = 1$, so what you see is that $BQ = 1 \text{ mod }P$, and therefore $B$ is an inverse of $Q$ in $K[x]/(P)$.
\end{example}

\section{Algebraic elements. Minimal polynomial}

So let me continue with my third example. 
\begin{example} \textit{(continued)}
\textbf{Way 2.} Instead of writing down the Bézout equality, one can also say that an ideal generated by an irreducible element of $K[x]$, $(P)$, is a maximal ideal. And the quotient by a maximal ideal is always a field.

But of course the proof of this amounts to the same Bézout equality. How do you prove $(P)$ is a maximum ideal? You just consider some potentially bigger ideal, that is to say containing $P$ and some element $Q$ not belonging to the ideal generated by $P$. This $Q$ is going to be prime with $P$, you can write the Bézout equality and it is in this way that you see that such an ideal will be necessarily equal to the polynomial ring $K[x]$. 

So this is an extension of $K$ in an obvious way. Or it's an extension of $K$ because it's a $K$-algebra, hence an extension of $K$.
\end{example}

Here a more concrete example. 
\begin{example}
Let's take $K$ equal to the field of two elements. So $K=\mathbb{F}_2=\mathbb{Z}/2\mathbb{Z}=\{0,1\}$. We have $1 + 1=0$. Let us take $P = x^2 + x + 1$, this is an irreducible polynomial over $\mathbb{F}_2$. Then $K[x]/(P)$ is a field of four elements. $K[x]/(P) = \{0,1,\overbar{x},\overbar{x+1}\}$. $\overbar{x}$ is the class of $x$ modulo $P$. And you see that $\overbar{x} ^2 = -\overbar{x}-1$ since you know that $x^2+x+1 = 0$ in our field. 

Well, the characteristics of $\mathbb{F}_2$ is 2, so $-1 = 1$, so it's just $\overbar{x+1}$, in the same way $(\overbar{x+1})^2 = \overbar{x}$, and they are inverse of each other: $\overbar{x}\cdot \overbar{x+1} = 1$. 

So this is the structure of a field of four elements. The cardinality of $K[x]/(P)$ is $|K[x]/(P)|=4$. One writes then $K[x]/(P)=\mathbb{F}_4$. Well, this might be strange at the first sight, because we only know that $K[x]/(P)$ has four elements. And if you write $\mathbb{F}_4$ you somehow mean that there is only one field of four elements. Well, it is true, there is only one field of four elements. In fact, all finite fields of the same cardinality are isomorphic, and we will see it very shortly. 
\end{example}

\begin{definition}
Let $L$ be an extension, $K \subset L$, and $\alpha \in L$. $\alpha$ is \textbf{algebraic}, if $\exists P \in K[x]$ such that $P(\alpha)=0$. Otherwise $\alpha$ is \textbf{transcendental}.
\end{definition}

\begin{lemma} If $\alpha$ is algebraic then there exists a unique monic polynomial $P$ of minimal degree with this property. Such a polynomial is irreducible. Any other polynomial $Q$ such that $Q(\alpha)=0$ is divisible by $P$.
\end{lemma}
\begin{definition}
Such a $P$ is called the \textbf{minimal polynomial} of $\alpha$ over $K$. We write $P_{\text{min}}(\alpha,x)$.
\end{definition}

\begin{proof}
The proof is a direct consequence of definitions. We know that $K[x]$, the polynomial ring in one variable, is a principal ideal domain. And the polynomials vanishing in $\alpha$, $I=\{Q\in K[x]|Q(\alpha)=0\}$, certainly form an ideal. So this ideal is generated by one element $P$, $I=(P)$. $P$ is certainly unique up to a constant element of minimal degree in $I$. Furthermore, if $P$ was not irreducible, $P=Q\cdot R$, then $P(\alpha) = Q(\alpha) \cdot R(\alpha)$. So $Q(\alpha)$ or $R(\alpha)$ must be 0, which is a contradiction with the minimality to the degree. 
\end{proof}

\section{Algebraic elements. Algebraic extensions}

We introduce an important notation. 
\begin{definition}
Let $L$ be an extension of $K$, $K\subset L$, and $\alpha\in L$. We write $K(\alpha)$ for the smallest subfield of $L$ containing $K$ and $\alpha$. We write $K[\alpha]$ for the smallest subring or $K$-algebra containing $K$ and $\alpha$. 
\end{definition}

Well, let me give you an example. 
\begin{example}
Well, first of all let me say that $K[\alpha]$ is generated as a vector space over K by 1, $\alpha$, $\alpha^2$, $\cdots\text{, }\alpha^n$, $\cdots$. So, let me give you an example. $\mathbb{C} = \mathbb{R}(i)$ as a field. But also to $\mathbb{C} = \mathbb{R}[i]$ as a ring. Because any element $z\in\mathbb{C}$ is $z=x + iy$ where $x$ is a real part and $y$ is the imaginary part. So, of course this is a vector subspace generated by $1$ and $i$. 
\end{example}

And this is a general phenomenon. So:
\begin{proposition} 
\label{p1}
The following conditions are equivalent (TFAE): 
\begin{enumerate}
\item $\alpha$ is algebraic over $K$
\item $K[\alpha]$ as a ring is a finite dimensional vector space over $K$
\item $K[\alpha]$ as a ring is equal to $K(\alpha)$ as a field.
\end{enumerate}.  
\end{proposition}

\begin{proof}
$(1)\implies(2)$. If $\alpha$ is algebraic, then
\begin{equation}
\alpha^d+a_{d-1}\alpha^{d-1}+\cdots+ a_1\alpha +a_0=0\text{ where }a_i \in K.
\end{equation}
This is just $P(\alpha)$, where $P$ is a minimal polynomial. So, we see that 
\begin{equation}
\alpha ^d = - \sum_{i=0}^{d-1} a_i\alpha^i.
\end{equation}
So, you see that $\alpha^d$, $\alpha^{d+1}$ and so on are going to be linear combinations of the lower powers of $\alpha$. This implies of course that $K[\alpha]$ is generated by $1$, $\alpha$, $\cdots$, $\alpha^{d-1}$ over $K$. 

And in particular it is finite dimension. 

$(2)\implies(3)$. It is enough to prove that $K[\alpha]$ is a field. Since, of course, $K[\alpha]\subset K(\alpha)$. 

Well, how do you prove this? It is very simple. Well, I'll let $x\in K[\alpha]$. I want to show that this is invertible. Consider the multiplication by $x$ ($K[\alpha]\to K[\alpha]$). This is a homomorphism of vector spaces, this is an injection of vector spaces over $K$. But, we know from linear algebra that if you have an injection of finite dimensional vector spaces of the same dimension, then it is also a surjection. Since $K[\alpha]$ is final dimensional this is a surjection. So there exist a $y\in K[\alpha]$, such that $y\cdot x = 1$. So, $x$ is invertible and $K[\alpha]$ is a field. Of course, I have forgotten to say that $x$ was supposed to be nonzero in order to have the multiplication by $x$ injective. But, I guess everybody has understood. So, we were assuming $x\neq 0$ from the beginning.

$(3)\implies(1)$. So if $K[\alpha]$ is a field, then $\alpha$ is algebraic. This is maybe the easiest part. So, if $\alpha$ is not algebraic, then $\nexists P$ such that $P(\alpha)=0$. But what does this mean? This means that a natural homeomorphism, $i:K[x]\to L$, $P \mapsto P(\alpha)$ is injective. But, $K[x]$ is not a field. And the image of this homeomorphism is $\text{Im}(i)=K[\alpha]$ which is a field. So, this is a contradiction.
\end{proof}


\begin{definition}
$L$ an extension of $K$ is called \textbf{algebraic} over $K$, if $\forall \alpha\in L$, $\alpha$ is algebraic. 
\end{definition}

Some properties of those algebraic extensions:

\begin{proposition}
If $L$ is algebraic over $K$, any $K$-subalgebra $L'$ of $L$ is a field. 
\end{proposition}

\begin{proof}
Let's take $L'\subset L$ a subalgebra. Fix an $\alpha \in L'$. I have to show that it's invertible. Well, I know it's algebraic. And then I know that $K[\alpha]$ -- which is also subalgebra of $L$ -- is a field. And then I know that $\alpha$ is invertible. And since I can do it for any $\alpha \neq 0$, it follows from this, that $L'$ is a field. 
\end{proof}

Well, another proposition, which will be important is as follows: 

\begin{proposition}
Let's have $K\subset L\subset M$. If $\alpha \in M$ is algebraic over $K$, then it's algebraic over $L$, and it's minimal polynomial over $L$ divides it's minimal polynomial over $K$. 
\[ P_{\text{min}}(\alpha,L) | P_{\text{min}}(\alpha,K) \]
\end{proposition}

\begin{proof} Well, this is of course clear since I can just consider $P_{\text{min}}(\alpha,K)$ as an element of $L[x]$.
\end{proof}

\section{Finite extensions. Algebraicity and fini-teness}

\begin{definition}
$L$ is said to be a finite extension of $K$, if it is a \textbf{finite dimensional} $K$-vector space. The dimension of $L$ over $K$ is called the \textbf{degree of the extension}. Notation: $\text{dim}_KL=[L:K]$.
\end{definition}

\begin{theorem}
\label{t1}
Suppose $L$ is an extension of $K$ and $M$ is an extension of $L$, $K\subset L\subset M$. Then $M$ is finite over $K$ if and only if $M$ is finite over $L$ and $L$ is finite over $K$. 

Moreover, in this case the degrees multiply. $[M:K]=[M:L]\cdot[L:K]$. 
\end{theorem}

\begin{proof}
Let me prove this theorem for you. One direction: suppose $M$ is finite over $K$. Well, any family $m_1$, $\cdots$, $m_n\in M$ which are linearly independent over $L$ of course is linearly independent over $K$. [This is obvious because what is linear independence of this is non-existence of linear combinations, of linear combinations with coefficients from $L$ or from $K$, which is zero. Of course, if there is no such combination with $L$ coefficients a fortiori, there is no such combination with $K$ coefficients because $K\subset L$.]

So, thus, the $\text{dim}_LM$ is finite because you cannot have a linear independent family of more than $\text{dim}_LM<\infty$. 

Well, now, on the other hand, $L$ is a $K$-vector subspace of $M$. So if $M$ is finite-dimensional over $K$, then $L$ is also finite dimensional over $K$ since the subspace of a finite dimensional vector space is also finite dimension. 

So one direction is easy. The other direction is also easy, but one must make a little computation. So let $e_1, ..., e_n$ be an $L$-basis of $M$ and let $\epsilon_1$, ..., 
$\epsilon_d$ be a $K$-basis of $L$. 

So, let us prove that $e_i \epsilon_j$ form a $K$-basis of $M$. Well indeed, $\forall x \in M$
is a linear combination of $e_i$ with $L$ coefficients. $x = \sum_{i=1}^n a_i e_i$, $a_i \in L$. Well, each $a_i$ is also a linear combination. Let's say, $a_i=\sum_{j=1}^d b_{ij} \epsilon_j$,  $b_{ij}\in K$. So we can write $x = \sum_{i, j} b_{ij} \epsilon_j e_i$. That is to say that $\epsilon_j e_i=e_i epsilon_j$, generate M over K. 

And we only have to check that these are linearly independent over $K$. If $\sum_{i, j} c_{ij} e_i \epsilon_j = 0$, then of course we can recompose the terms. Then we have $\sum_i (\sum_j c_{ij} \epsilon_j)e_i  = 0$, where $\sum_j c_{ij} \epsilon_j\in L$. But $e_i$ form a basis, so $\forall i$, $\sum_j c_{ij} \epsilon_j=0$. But this means -- as $\epsilon_j$ form a basis -- that $c_{ij}=0$, $\forall i,j$. So the theorem is proved. 
\end{proof}

\begin{definition}
Notation: Let $K(\alpha_1, \cdots, \alpha_n) \subset L$ be the smallest subfield of $L$ containing $K$ and the $\alpha_i$. I will also often say that this is generated by $\alpha_1, ..., \alpha_n$ over $K$. 
\end{definition}

\begin{theorem}
$L$ is finite over $K$ if and only if $L$ is generated by a finite number of algebraic elements over $K$.
\end{theorem}
\begin{proof}
Well again, one direction is obvious. If $L$ is a finite dimensional $K$-vector space, then we can take a $K$-basis $\alpha_1, \cdots, \alpha_d$. Then $L$ is certainly equal to the smallest subring of $L$ containing $\alpha_1,\cdots, \alpha_d$, which is a field, so it is also a smallest subfield of L containing $\alpha_i$. 
\begin{equation}
L=K[\alpha_1,\cdots, \alpha_d]=K(\alpha_1,\cdots, \alpha_d).
\end{equation}

Now, all $\alpha_i$ are algebraic, this is just because, well moreover, each $K[\alpha_i]$ is a finite dimensional K-algebra, since it is a subring of L which is already finite-dimensional. So then by proposition \ref{p1}, $\alpha_i$ is algebraic. 

Well, in the other direction it is also not difficult. We have $K[\alpha_1]$ is finite dimensional over K, $K[\alpha_1, \alpha_2]$ is finite dimensional over $K[\alpha_1]$, and so on. In general $K[\alpha_1, \cdots, \alpha_d]$ is finite dimensional over $K[\alpha_1, \cdots, \alpha_{d-1}]$. 

All of those elements are algebraic so all $K[\alpha_1, ..., \alpha_i]$ are fields.
So $K[\alpha_1, ..., \alpha_i]=K(\alpha_1, ..., \alpha_i)$. And now, just use theorem \ref{t1} to conclude that $L$ which is $K(\alpha_1, ..., \alpha_d)$ is finite over $K$.
\end{proof}

\section{Algebraicity in towers. An example.}
Algebraic extensions satisfy a similar property to that of finite extensions. If you have a tower of algebraic extensions, then it is algebraic if and only if  each floor of this tower is algebraic. 

\begin{theorem}
\label{t2}
Well, again let $L$ be an extension $K$, and $M$ an extension of $L$, $K\subset L\subset M$. Then $M$ is algebraic over $K$ if and only if $M$ is algebraic over $L$ and $L$ is algebraic over $K$.
\end{theorem}

\begin{proof}
So, one direction: let $\alpha\in M$. Of course, if it satisfies a polynomial relation with coefficients from $K$, if $P(\alpha) = 0$, for some $P \in K[x]$ then also, this $P \in L[x]$. So $\alpha$ is algebraic over $L$. And if now $\alpha \in L$ -- we want to prove that it's algebraic over $K$-- then also $\alpha \in M$, and so $\alpha$ is algebraic over $K$. And so $L$ is algebraic over $K$. 

Another direction. So, I have $L$, which is algebraic over $K$ and $M$ algebraic over $L$. I must prove that $M$ is algebraic over $K$. So, take $\alpha \in M$ and consider the minimal polynomial $P_{\text{min}}(\alpha,L)$. Its coefficients are elements of $L$, so they are algebraic over $K$. By the previous theorem, theorem \ref{t2}, they generate an extension $E$ which is finite over $K$. Now, $E(\alpha)$ is also finite over $K$ since $E(\alpha)$ is finite over $E$. And this means that $\alpha$ is algebraic over $K$ because -- since $E(\alpha)$ is finite over $K$ -- there will be some linear dependence between the powers of $\alpha$. 
\end{proof}

Okay, let me give you an example. 
\begin{example}
So consider the extension obtained from $\mathbb{Q}$ by adjoining, let's say, $\sqrt[3]{2}$ and or $\sqrt{3}$, that is $\mathbb{Q}(\sqrt[3]{2},\sqrt{3})$. This is clearly algebraic and finite over $\mathbb{Q}$. And what is it's degree? The degree of this extension is 6, indeed, we have
\begin{equation}
\mathbb{Q} \subset \mathbb{Q}(\sqrt[3]{2}) \subset \mathbb{Q}(\sqrt[3]{2},\sqrt{3}).
\end{equation}
$P_{\text{min}}(\sqrt[3]{2},\mathbb{Q})=x^3-2$. This is an irreducible polynomial over $\mathbb{Q}$, which has the $\sqrt[3]{2}$ as a root. And so $\mathbb{Q}(\sqrt[3]{2})$ is generated over $\mathbb{Q}$ by linearly independent elements: $1$, $\sqrt[3]{2}$, and $(\sqrt[3]{2})^2$. So, the degree of $\mathbb{Q}(\sqrt[3]{2})$ over $\mathbb{Q}$ is equal to $3$, $\left[\mathbb{Q}(\sqrt[3]{2}):\mathbb{Q}\right]=3$.

Now, $\sqrt{3} \notin \mathbb{Q}(\sqrt[3]{2})$. Well, maybe the easiest way to see it is as follows: because otherwise, one would have $\mathbb{Q} \subset \mathbb{Q}(\sqrt{3}) \subset \mathbb{Q}(\sqrt[3]{2})$. But $\left[\mathbb{Q}(\sqrt{3}):\mathbb{Q}\right]=2$ has to divides $\left[\mathbb{Q}(\sqrt[3]{2}):\mathbb{Q}\right]=3$. (We have seen that the degrees multiply in towers of finite extensions.) So this is impossible. Therefore $x^2 -3$ is irreducible over our extension $\mathbb{Q}(\sqrt[3]{2})$, and $x^2 -3= P_{\text{min}}(\sqrt{3},\mathbb{Q}(\sqrt[3]{2}))$. 

And the degree of our big extension $\left[\mathbb{Q}(\sqrt[3]{2},\sqrt{3}):\mathbb{Q}(\sqrt[3]{2})\right]=2$. Therefore $\left[\mathbb{Q}(\sqrt[3]{2},\sqrt{3}):\mathbb{Q}\right]=2 \cdot 3 = 6$. 
\end{example}

Well, I should have said this earlier. This fact is completely general that the degree of an extension generated by an algebraic element $\alpha$ over $K$ is equal to the degree of the minimal polynomial of $\alpha$. 

\begin{proposition}
$\left[K(\alpha):K\right]=\text{deg}P_{\text{min}}(\alpha,K)$ if $\alpha$ is algebraic.
\end{proposition}
\begin{proof}
I should have said this earlier, that the proof is obvious. Since we have already remarked several times that $K(\alpha)$ is generated by $1, \alpha, \cdots, \alpha^{d-1}$, where $d=\text{deg}P_{\text{min}}(\alpha,K)$. And in fact this is a basis, independent. 
\end{proof}

Well in any case this is a good practical tool to compute the degree of finite extension. And I would like to finish with the little remark which might be helpful to understand the next lecture. So a proposition: 

\begin{proposition}
Let $K \subset L$ be a field extension. So consider $L'\subset L$, formed by all elements algebraic over $K$. Then $L'$ is a subfield of $L$. 
\end{proposition}

\begin{definition}
One calls it the \textit{algebraic closure} of $K$ in $L$.
\end{definition}

\begin{proof}
Well, the proof is, of course, easy. If $\alpha$ and $\beta$ are algebraic over $K$, we have to prove that $\alpha + \beta$ and $\alpha \cdot \beta$ are algebraic. But this is trivial by theorem \ref{t2}, since  $\alpha + \beta, \alpha \cdot \beta \in K[\alpha, \beta]$ which is a finite (by theorem \ref{t2}) extension of $K$.
\end{proof}


\section{A digression: Gauss lemma, Eisenstein criterion.}

Let me summarize what we have done up till now. 
\begin{itemize}
\item $K$ field, then $\alpha$ is algebraic over $K$, if root of $P\in K[x]$ (K[x] are the polynomials with coefficients in $K$).
\item $L$ algebraic over $K$, if $\forall \alpha \in L$ algebraic over $K$. 
\item $F$ finite over $K$, if $\text{dim}_KL<\infty$ (finite dimensional $K$-vector space). 
\item Finite $\implies$ algebraic. 
\item Finite $\iff$ algebraic and finitely generated. 
\item $\left[K(\alpha):K\right]=\text{deg}P_{\text{min}}(\alpha, K)$ (We generate $L$ by a single element $\alpha$)
\end{itemize}

So, given some algebraic element over $K$, a root of some polynomial, it is important to be able to decide whether this is the minimal polynomial of $\alpha$ over $K$. That is to say, it is important to have some \textit{irreducibility criteria}, and this is something you probably already know. But since it's so important, I would like to remind a couple of things about this.

So how to decide, that a polynomial is irreducible over $K$? Well in our example, we had a very simple polynomial. So $x^3-2$ was irreducible over $\mathbb{Q}$ since it's a cubic polynomial, so if it was not irreducible it would have a root in $\mathbb{Q}$. So this is easy since the degree is equal to $3$ and there is no root. But, well, if you ask the question whether $x^{100}-2$ is irreducible or not, this is already not so simple, right? Well, this is irreducible. And here are a couple of facts which help to see this easily. 

\textbf{Fact 1 (Gauss Lemma).}  If $P$ decomposes nontrivially. By this I mean $P$ is a product of two factors of strictly smaller degree over $\mathbb{Q}$ (that is $P = Q \cdot R$, where $\text{deg}Q\text{, }\text{deg}R<\text{deg}P$, or the same is to say that $0<\text{deg}Q\text{, }\text{deg}R$), then it is also the case over $\mathbb{Z}$. Well, of course I have to say that I am considering a polynomial with integral coefficients. Well, let me give a proof. So, over $\mathbb{Q}$ write $P =Q \cdot R$. They are not integral, but of course you can multiply by a common denominator, and they become integral. So lets say $mQ=Q_1\in \mathbb{Z}[x]$ and $nR=R_1\in \mathbb{Z}[x]$. Then we have $mnP=Q_1 R_1$ over $\mathbb{Z}[x]$. Then take $p|mn$. Then modulo $p$ (means over $\mathbb{F}_p$) we have $0 = \overbar{Q_1} \cdot \overbar{R_1}$ (where bar denotes the reduction modulo $p$) But we are over $\mathbb{Z}/p\mathbb{Z}$, which is a field, so we have either $\overbar{Q_1}=0$ or $\overbar{R_1}=0$. This means that $p$ divides all coefficients either of $Q_1$ or of $R_1$. Say, of $Q_1$, then we can write $\frac{mn}{p}\cdot P=Q_2\cdot R_1$ in $\mathbb{Z}[x]$, and here of course $Q_2 = Q_1 / p$. Continuing in this way, We arrive at, I don't know, $P=Q_l R_s$ in $\mathbb{Z}[x]$. 

\textbf{Fact 2 (Eisenstein criteria).} Let me not prove it in full generality, let me just show this on an example, and then formulate maybe a general criteria. Well how to show that $X^{100}-2$ is irreducible over $\mathbb{Z}$? This is very easy. We reduce modulo $2$, and so if $x^{100} - 2$ decomposes nontrivially as $Q \cdot R\text{ mod }2$ then $x^{100}=\overbar{Q}\cdot\overbar{R}$ in  $\mathbb{F}_2[x]$. This means that $\overbar{Q}$ and $\overbar{R}$ are of the form $x^k$, respectively $x^l$. So they have no constant coefficient modulo two. So the constant coefficient, constant coefficient of both $\overbar{Q}$ and $\overbar{R}$ is even and this means the constant coefficient of $x^{100} - 2$ must be divisible by $4$. And this is not the case.

So the general formulation of Eisenstein's criterion criterion is as follows: In general, if you have a polynomial with integral coefficients $P =a_n x^n + a_{n-1} x^{n-1} + \cdots + a_0 \in \mathbb{Z}[x]$. If $\exists p$ prime such that $p\nmid a_n$, $p|a_i$ for $i\neq n$ and $p^2\nmid a_0$, then $P\in\mathbb{Z}[x]$ is irreducible. And the proof is exactly the same. 

And to conclude, I would like to say that both facts are valid replacing $\mathbb{Z}$ by some unique factorization domain $R$ and replacing $\mathbb{Q}$ by it's fraction field. So, I think this is a good point to finish the first lecture.

