\chapter{Stem field, splitting field, algebraic closure}

\section{Stem field. Some irreducibility criteria}
Consider a polynomial $P \in K[x]$, which is irreducible and monic. Let me give the definition

\begin{definition}
A \textit{stem field} for $P$ is an extension $E$ of $K$, such that $\alpha\in E$ root of $P$, and $E$ is generated by this root $E=K[\alpha]$.
\end{definition}

Well, such a thing exists: we can take just $K[x]/(P)$. This is a field, since $P$ is irreducible. Well, on the other hand, any stem field $E$ is isomorphic to such a thing: $E\cong K[x]/(P)$. While it is easier to define the isomorphism in the other direction. Well, let's see, $K[x]/(P)\to E$ by $f\mapsto f(\alpha)$. 

Okay. To summarize, we have the following proposition: 
\begin{proposition}
A stem field exists and if $E$ and $E'$ are two stem fields for $P\in K[\alpha]$, $E=K[\alpha]$, $E'=K[\alpha']$ ( $\alpha$ and $\alpha'$ are roots of $P$), then there exists a unique isomorphism  ($K$-algebras) $E\to E'$, $\alpha\mapsto\alpha'$.
\end{proposition}

\begin{proof}
So existence we have already seen. Uniqueness of the isomorphism: Isomorphism of $K[\alpha]$ with $E'$ is defined by its value on $\alpha$.

So, what I have to prove still is the existence of such an isomorphism. But this is easy, because I have $\phi: K[x]/(P)\to E$, $x\mapsto\alpha$, and $\psi: K[x]/(P) \to E'$, $x\mapsto\alpha'$. And I just take $\psi^{-1} \cdot \phi: E\cong E'$, $\alpha \mapsto \alpha'$. 
\end{proof}

Remarks: 
\begin{enumerate}
\item In particular if a stem field contains two roots of $P$, then there exists a unique automorphism of it, taking one to the other. 
\item If $E$ is a stem field of an irreducible polynomial, then $[E:K]=\text{deg }P$. And conversely: If $[E:K]=\text{deg }P$ and $E$ contains a root of $P$ then $E$ is a stem field. (Otherwise it's degree would be strictly greater then the degree of P.)
\end{enumerate}

Now I can give you some more irreducible criteria.
\begin{corollary}
$P\in K[x]$ irreducible over $K$ if and only if it does not have roots in extensions L of K of degree $\leq n/2$, where $n=\text{deg }P$.
\end{corollary}

\begin{proof}
Otherwise if $P$ is not irreducible, then it has a prime, irreducible factor $Q$  with $\text{deg }Q\leq n/2$ (where $n=\text{deg }P$) and one can take it's stem field as $L$.

In the other direction. Conversely, if $P$ has a root $\alpha \in L$, then of course $P_{\text{min}}(\alpha,K)$ divides $P$. So $P$ cannot be irreducible. 
\end{proof}

Okay, one more irreducibility criterion. 
\begin{corollary}
Let $P\in K[x]$ irreducible, $\text{deg }P=n$. Let $L$ be an extension $\text{deg }L=m$. If $(n,m)=1$ (n and m are relatively prime), then $P$ is irreducible over $L$.
\end{corollary}

Okay, you see, an irreducible polynomial might become reducible over an extension, but it doesn't happen if the degrees are relatively prime. 

\begin{proof}
If this is not the case $Q | P$ in $L[x]$, let $M$ be a stem field of $Q$ over $L$. So, we have $K\subset L\subset M=L[\alpha]$. So $K[\alpha]$ ($\alpha$ is a root of $Q$ so it's a root of $P$), is a stem field of $P$ over $K$. So $\text{deg}_KK(\alpha)=n=\text{deg}P$. 

Well, on the other hand, if $\text{deg}Q=d$, then $[M:L]=d$. And so the total degree $[M:K]=m\cdot d$. But $K[\alpha]$ is a subextension of $M$. So the degree $n|m\cdot d$. And since $(n,m)=1$, then, $n | d$, but $d<n$. So, $n = d$ and $P$ is irreducible over L. 
\end{proof}


\section{Splitting field}
$P\in K[x]$ not necessarily irreducible. 
\begin{definition}
A \textit{splitting field} of $P$ over $K$ is an extension $L$, where $P$ is split (i.e. is a product of linear factors) and the roots of $P$ generate $L$. 
\end{definition}
It's the smallest field extension where $P$ is split. 

\begin{theorem}
\begin{enumerate}
\item A splitting field exists and its degree over $K$ is less or equal than $d!$, where $d=\text{deg}P$.
\item (Uniqueness up to an isomorphism.) If $L$, $L'$ are two splitting fields then they are isomorphic as K-algebras. But such an isomorphism will not necessarily be unique.
\end{enumerate}
\end{theorem}

\begin{proof}
We shall prove this theorem by induction on $d$. If $d=1$ everything is trivial. And a splitting field is just $K$ itself. So, let me suppose that $d>1$, and theorem is proved for all polynomials of degree $<d$ and over any field $K$. Then we take $Q$ an irreducible factor of $P$. Let $\alpha$ be a root. So $L_1=K[\alpha]$ is a stem field of $Q$. So, over $L_1$ we have $P=(x - \alpha)R$. And we know that we have a splitting field $L$ of $R$ over $L_1$ and its degree is at most $\text{deg}R!\leq (d-1)$. This will be also a splitting field of P over K. $[L:K] = [L:L_1]\cdot[L_1:K]\leq (d-a)! \cdot d= d!$ 

Now it remains to prove uniqueness up to isomorphism. Let $L$ and $M$ be two splitting fields. Now, let $\beta$ be a root of $Q$ (some irreducible factor of $P$) in $M$.  Then $K[\alpha]=L_1$ and and $K[\beta]$ are both stem fields for $Q$. So, we have an isomorphism $\phi: K[\alpha] \to K[\beta]$, $\alpha \mapsto \beta$.

Now $P=(x-\beta) \cdot S$ in $M[x]$. So, where $S=\phi(R)$. So, $M$ is a splitting field of $S$ over $K[\beta]$. M is an extension of $K[\beta]$. But now let's remember that we have also defined a field extension as an algebra. But $M$ is also a $K[\alpha]$-algebra via $\phi$. And as such it is a splitting field of R over $K[\alpha]$. Well, I know you have to meditate about this a little bit, but this is true as soon as you view $M$ as a $K$-algebra. You sort of take this $\phi$ into account. So by induction we have $K[\alpha]$-isomorphism $L \to M$. And of course we also have a $K$-isomorphism. 
\end{proof}

Remark: The isomorphism between two splitting fields is not unique. A splitting field in particular can have many $K$-automorphisms. And in fact, the subject of Galois theory is to study this group of automorphisms. Which you will see in a couple of lectures.

\section{An example. Algebraic closure}
\begin{example}
Let's take the same polynomial $x^3 - 2$ over $\mathbb{Q}$. Its roots are: $\sqrt[3]{2}$, $j\sqrt[3]{2}$, and $j^2\sqrt[3]{2}$, where $j=e^{2 \pi i/3}$. The splitting field $L=\mathbb{Q}(\sqrt[3]{2},j)$. 

Now let us find the automorphisms of $L$. Well, for these let me write two towers. $\mathbb{Q}\subset \mathbb{Q}(j)$ (degree 2) and $\mathbb{Q}\subset \mathbb{Q}(\sqrt[3]{2})$ (degree 3). 

And $\mathbb{Q}(j)\subset \mathbb{Q}(j,\sqrt[3]{2})$ (degree 3) and $\mathbb{Q}(\sqrt[3]{2})\subset \mathbb{Q}(j,\sqrt[3]{2})$ (degree 2).

We see that there is a $\mathbb{Q}(j)$-automorphism $\sigma$ of $L$ taking $\sqrt[3]{2}\mapsto j\sqrt[3]{2}$. Because $L$ is a stem field of $x^3 - 2$ over $\mathbb{Q}(j)$, so there are automorphisms interchanging roots. And there is also a $\mathbb{Q}(\sqrt[3]{2})$-automorphism $\tau$ of $L$ taking $j\mapsto j^2$. Since these are two roots of the same minimal polynomial. So $L$ is a stem field of the minimal polynomial over $\mathbb{Q}$ of $\sqrt[3]{2}$, and there is an automorphism exchanging those roots. So we have a whole group of automorphisms, which is in fact equal to the group of permutations on three elements. Well, the group of automorphisms of $L$ over $K$ is embedded into $S_3$ (the permutation groups of 3 elements). Because of those automorphisms permute the roots of $x^3 - 2$. In fact, $\text{Aut}(L/K)=S_3$ since those $\sigma$ and $\tau$ already generate $S_3$. 
\end{example}

Let me now talk about algebraic closure. 

\begin{definition}
A field $K$ is \textit{algebraically closed}, if any nonconstant polynomial has a root in $K$.
\end{definition}

This is the same as to say that any nonconstant polynomial splits as a product of linear factors. 

\begin{example}
For example the field of complex numbers has this property. In fact, we will give a proof of this in the future. Shall be proved by almost pure algebraic means.
\end{example}

\begin{definition}
An \textit{algebraic closure} of $K$ is a field $L$, which is algebraically closed and algebraic over $K$.
\end{definition}

\begin{theorem}
Any field $K$ has an algebraic closure.
\end{theorem}

Note that I don't say at this point anything about uniquness. Up to isomorphism or whatever. We will treat this later. Now, let's concentrate on existence. 

\begin{proof}
Proof is a bit weird. How are we going to proceed? So we first construct $K_1$ such that $\forall P \in K[x]$ has a root in $K_1$. Well, this is not yet a victory because we don't know whether any polynomial with coefficients in $K_1$ has a root in $K_1$. Maybe we have introduced some new rootless polynomials. Then construct $K_2$ such that $\forall P\in K_1[x]$ has a root in $K_2$, and so on and so forth. This is the stratagy. 

$K \subset K_1 \subset K_2 \cdots \subset K_n\subset\cdots$. Take $\overbar{K}=\bigcup K_n$. I claim that $\overbar{K}$ is algebraically closed. Indeed, $\forall P\in \overbar{K}[x]$, $\exists n$: $P\in K_n[x]$. So it has root in $K_{n+1}$, so in $\overbar{K}$. So, if we learn to construct such $K_1, K_2$ and so on this will solve the problem. 
\end{proof}


\section{Algebraic closure (continued)}

\begin{proof} \textit{continued.} 
Construction of $K_1$. Let $S$ be the set of all irreducible elements of $K[x]$. So $S$ is a very big set. Let $A=K[(X_P)_{P\in S}]$ (One variable $X_P$ for every $P \in S$). Very big polynomial ring. Let $I\in A$ be the ideal generated by all $P(X_P)$, $\forall P \in S$. So what I claim, is that $I\neq A$ is a proper ideal. Well, indeed if not, then I can write $1\in A$ as $1=\sum \lambda_i P_i(X_{P_i})$. So these some generators of $I$. These are elements of $A$, right? So also some crazy polynomials in crazy variables. But the main point is that this sum is finite. If my ideal contains $1$, then $1$ is a finite, a linear combination of my generators. Okay. Well, take $L$ a splitting field of $\prod_{i=1}^{n}P_i$ over $K$. I generate an extension of K by all the roots of all my $P_i$ which were irreducible over $K$, okay? So let $\alpha_i\in K$ be a root of $P_i$ in K. Then, my $A$ is polynomial ring. And it's very easy to produce a homomorphism of a polynomial algebra to some other algebra. One must just note where one sends the variables. So, $\exists \phi: A \to L$, $X_{P_i}\mapsto \alpha_i$ and $X_{P}\mapsto 0$ if $P\neq P_i$. (My A is a polynomial algebra.)

Then $\phi(1)=0$.  And $\phi(P_i(X_{P_i}))=P_i(\alpha_i)=0$. But this cannot happen. $\phi(1)$ must be equal to $1$. So, $I$ is an ideal. Then you know how to get a field.  So a fact: any ideal, any proper ideal in any commutative associative ring with unity is contained in a maximal ideal $m$ and the quotient $A/m$ is a field. So I just can take $K_1 = A/m$. And then continue in the same way to construct $K_2$, ... $K_3$, $K_n$ and so forth. Maybe I should give you some comment on this fact. This is very important, and
the technique is very important.

Digression: ideals in a ring. Any proper ideal is contained in a maximal ideal. This is a consequence of what one calls Zorn's lemma, So let me give you the Zorn's lemma. Consider P, not a polynomial anymore, but a partially ordered set. $C\subset P$ is called a chain if it is totally ordered, that is $\forall \alpha , \beta \in C$, $\alpha \leq \beta$ or $\beta \leq \alpha$.  Where $\leq$ is our order relation. Zorn's lemma says: if any non-empty chain in a non-empty $P$ has an upper bound (that is an $M\in P$ such that $M\geq x \forall x \in C$, then $P$ has maximal elements. I will not of course prove this Zorn's lemma because you certainly have heard that this is the same thing basically as the axiom of choice or Zermelo's theorem, so this is relevant for set theory, foundations of mathematics, not to algebra and Galois Theory. But, we shall use it to prove that any ideal is contained in a maximal ideal. 

Now, let $P$ be a set of proper ideals in $A$ containing $I$. This is non empty because contains $I$. 
And any chain $\left\{ I_{\alpha} \right\}_{\alpha\in J}$ has an upper bound. 
This is just the $\bigcup I_{\alpha}$. You check that it is an ideal. I'll leave it as an exercise. So, our set $P$ has maximal elements. So, $I \subset M$, $M$ maximal ideal. 

And if we have a maximal ideal, if we take a quotient by a maximal ideal, then this is certainly a field. Well, otherwise it would have some proper ideals $a \in A/m$ would generate a proper ideal. And it's preimage under the projection from A to A/m would strictly contain $m$.\\
\end{proof}


\section{Extension of homomorphisms. Uniqueness of algebraic closure}
To sum up we have just proved the existence of an algebraic closure $\overbar{K}=\bigcup_{i=1}^{\infty}K_i$. The set of $K_i$'s was a chain ($\cdots\subset K_i\subset K_{i+1}\subset \cdots$), and each $K_i$ was constructed as follows: It is a field where each $P\in K_{i-1}[x]$ has a root. And we constructed these $K_i$ by considering a huge polynomial ring over $K_{i-1}$ by a suitable maximal ideal. And to construct a maximal ideal we first have constructed a proper ideal and then have used Zorn's lemma to derive a maximal ideal. 

Is there any uniqueness result for the algebraic closure? Yes, we have such a result and to prove this I have to prove another theorem. 

\begin{theorem}[Extension of homomorphisms]
Let take $K \subset L \subset M$ algebraic extensions. $K$ embended in some algebraic closure, $K\subset \Omega$. Then any homomorphism $\phi:L\to \Omega$ extends to a homomorphism $\widetilde{\phi}: M \to \Omega$.
\end{theorem}

\begin{proof}
And the proof is again an application of Zorn's lemma. Apply Zorn to the following set $\xi=\left\{ (N,\psi): L\subset N\subset M \text{, }\psi\text{ extends }\phi\right\}$. So $\xi$ is not empty. Because for instance $(L, \phi)\in \xi$. What is the order? Partially ordered by the following relation: $(N, \psi)\leq (N', \psi')$ if $N \subseteq N'$ (embedded) and $\psi'$ extends $\psi$ ($\psi'|_N=\psi$).

Now, any chain $(N_{\alpha}, \psi_{\alpha})$ has an upper bound. $\bigcup_{\alpha}N_{\alpha}$ is a subextension of $M$. $\phi$ defined in the obvious way (for $x \in N_{\alpha}$, $\psi(x)=\psi_{\alpha}(x)$). (We take $(N, \psi)$ as an upper bound for our chain.)

So our set $\xi$ has maximal elements. So let $(N_0,\psi_0)$ be one of them. And suppose that $N_0\neq M$ (strictly included in $M$). Well, now it's very easy to get a contradiction. Take $x \in M\setminus N_0$  and consider $P_{\text{min}}(x,N_0)$. Let $\alpha$ be a root in $\Omega$ (algebraic closure). Then, define $N_0(x)\to \Omega$ by $x \mapsto \alpha$ and $\psi_0$ on $N_0$. So this gives a contradiction with maximality of $(N_0, \psi_0)$. But we have found a way to extend it again. So we've got a contradiction, so $N_0 = M$ take $\widetilde{\phi} = \psi_0$.
\end{proof}

\begin{corollary}
The algebraic closure is unique up to an isomorphism. If $\Omega, \Omega'$ are algebraic closures of $K$ then they are isomorphic as $K$-algebras.
\end{corollary}

\begin{proof}
Since, you have an imbedding of $K$ into $\Omega$ and to $\Omega'$. Then you extend $i$ to a map of $\phi: \Omega' \to \Omega$. And, in fact, since you can do it in the other direction as well, you eventually obtain that such a map must be an isomorphous. 
\end{proof} 

\section{An example (of extension)}

Let me formulate two corollaries of the theorem on extension of homomorphisms. 

\begin{corollary}
An algebraic closure of $K$ is unique up to an isomorphism of $K$-algebras.
\end{corollary}
\begin{corollary}
Any algebraic extension of $K$ embeds into the algebraic closure. 
\end{corollary}

\begin{example} \textit{Extension of homomorphisms.} Let's take $K = \mathbb{Q}$. And let's fix an algebraic closure, $\overbar{\mathbb{Q}}$ (e.g. $\overbar{\mathbb{Q}}\subset\mathbb{C}$, the set of algebraic numbers. That is roots of polynomials with rational coefficients ). 

$L=\mathbb{Q}(\sqrt{2})=\mathbb{Q}[x]/(x^2-2)$. Let $\alpha$ denote the class of $x\in L$. And $L$ has two embeddings in $\overbar{\mathbb{Q}}$. $\phi_1:\alpha\mapsto\sqrt{2}$ and $\phi_2:\alpha\mapsto-\sqrt{2}$. $\phi_i$ are identity on $\mathbb{Q}$. $\alpha$ can go to both roots of the polynomial, $x^2 - 2$.

Now, consider $M=\mathbb{Q}(\sqrt[4]{2})=\mathbb{Q}/(y^4 - 2)$. Let $\beta$ denote the class of $x\in M$. $M$ hat $4$ embedding in $\overbar{\mathbb{Q}}$. That is $\beta\mapsto \pm \sqrt[4]{2}, \pm i\sqrt[4]{2}$. $\psi_1:\beta\mapsto\sqrt[4]{2}$ and $\psi_2:\beta\mapsto-\sqrt[4]{2}$ extend $\phi_1$. $M$ is an extension of $L$, $M=L[y]/(y^2-\alpha)$.

$\psi_3:\beta\mapsto-i\sqrt[4]{2}$ and $\psi_4:\beta\mapsto i\sqrt[4]{2}$ extend $\phi_2$. 
$\pm i\sqrt[4]{2}$ are the square roots of $ -\sqrt{2}$.
\end{example}
