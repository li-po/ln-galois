\chapter{Finite fields. Separability, perfect fields}

\section{Finite fields}

We have seen: $K$ finite field $\implies$ $\text{char}K=p$, $p$ prime number. $K$ finite extension of $\mathbb{F}_p$, which is a finite dimensional vector space over $\mathbb{F}_p$. If $n=[K:\mathbb{F}_p]$, then the cardinality of K is $|K|=p^n$. Notation: $K =\mathbb{F}_{p^n}$. 

Does $K$ exists? Is $K$ unique?

\textbf{Remark.} If $\text{char }K=p$, then consider $F_p:K\to K$, $x\mapsto x^p$. This is a field homomorphism: $(x + y)^p= x^p + y^p$ and $(xy)^p = x^p y^p$. 

The second properties is true in all fields and the first is particular for characteristic p. Because if you decompose this $(x + y)^p$ by the binomial formula, almost all coefficients will be divisible by $p$.  It is called \textit{Frobenius homomorphism}.

In the same way, $F_{p^n}: x \mapsto x^{p^n}$ is also a field homomorphism.

\begin{theorem}
Fix an algebraic closure $\mathbb{F}_p\subset\overbar{\mathbb{F}_p}$. A splitting field of $x^{p^n} - x$, the field generated by its roots in $\overbar{\mathbb{F}_p}$, has $p^n$ elements. Conversely: Any field of $p^n$ elements is a splitting field of $x^{p^n} - x$. Moreover, there is a unique subextension of $\overbar{\mathbb{F}_p}$ consisting of $p^n$ elements.
\end{theorem}


\section{Properties of finite fields}

\begin{proof}
So recall that we have seen that the map $F_{p^n}:x \mapsto x^{p^n}$ is a homomorphism. $\implies$ $\left\{ x|F_{p^n}(x)=x\right\}$ is a subfield (it contains $F_p$, it contains the roots of $Q_n(x)=X^{p^n}-x$).

This subfield is exactly the splitting field of $Q_n$, and since $Q_n$ does not have multiple roots (this can be seen by verifying that $Q_n$ is relatively prime with its derivative $Q'_n=1$) $\implies$ there are  exactly $p^n$ roots $\implies$ this is a splitting field of $Q_n$ $=$ the roots of $Q_n$ $=$ field of $p^n$ elements. 

Conversely, we show that any field of $p^n$ elements is a splitting field of $Q_n$. Let $|K|=p^n$ and $\alpha \in K$ then $\alpha^{p^n - 1}=$ (provided that $\alpha \neq 0$). Indeed, the multiplicative group of $K$ has cardinality $|K^*|=p^n - 1$. $\alpha$ is a root of $x^{p^n} - x$ and $0$ is also a root $\implies$ $K$ consist of roots of $Q_n$. This also answers the question why there is a unique subextension of $\overbar{F_p}$, which has $p^n$ elements. This is just because these subextension consists exactly of the roots of $Q_n$. The unicity of the subextension (of the image of the embedding of $K$ into $\overbar{F_p}$) also follows. 
\end{proof}

\begin{theorem}
$\mathbb{F}_{p^n}\supset \mathbb{F}_{p^d} \iff d|n$.
\end{theorem}
	
\begin{proof}
$\Rightarrow$ is the multiplicativity of degrees in towers. $n=[\mathbb{F}_{p^n} :\mathbb{F}_p]=[\mathbb{F}_{p^n} :\mathbb{F}_{p^d}]\cdot[\mathbb{F}_{p^d} :\mathbb{F}_p]$, $[\mathbb{F}_{p^d} :\mathbb{F}_p]=d$.

$\Leftarrow$: Conversely, suppose that $d|n$, then if $x^{p^d}=x$, the same is true for $x^{p^n}=x$. $\implies$ $\mathbb{F}_{p^d}\subset \mathbb{F}_{p^n}$.
\end{proof}
 
\begin{theorem}
$\mathbb{F}_{p^n}$ is a stem field and a splitting field of any irreducible polynomial $P\in \mathbb{F}_p[x]$ of degree $n$.
\end{theorem}
	
\begin{proof}
Indeed a stem field of $P$ has degree $n$ over $\mathbb{F}_p$. So this is $F_{p^n}$. 

Let $\alpha$ be a root of $P$, $\alpha \in \mathbb{F}_{p^n}$, $\implies$ $Q_n(\alpha)=0$ (since $Q_n$ has as roots exactly all elements of $\mathbb{F}_{p^n}$). $P|Q_n$ $\implies$ $P$ splits in $\mathbb{F}_{p^n}$.
\end{proof}
 
\begin{corollary} 
$Q_n=\prod_{d|n}\prod_{P \text{ irred. monic and deg }P=d}P$.\end{corollary} 
\begin{proof}
We have already seen that all such $P$ divide $Q_n$ (since the stem field is $\mathbb{F}_{p^d}\subset \mathbb{F}_{p^n}$, so $Q_n(\alpha)=0$ if $\alpha$ is a root). $\implies$ $\prod_{d|n}\prod_{P:*(P,d)}P|Q_n$,with the condition $*(P,d)=P\text{ irreducible monic and deg }P=d$. $Q_n$ has no multiple roots $\implies$ there are no multiple factors. 

What remains to prove is that there are no other irreducible factors of $Q_n$. So let $R$ be an irreducible factor of $Q_n$. Then if $\alpha$ is a root of $R$, $Q_n(\alpha)=0$. So this means that $\mathbb{F}_p(\alpha)\subset \mathbb{F}_{p^n}$ $\implies$ $\mathbb{F}_p(\alpha) = \mathbb{F}_{p^d}$, $d|n$ $\implies$ $\text{deg }R|n$. That is there are no other irreducible factors.
\end{proof}

\section{Multiplicative group and automorphism group of a finite field}

The next theorem say that the multiplicative group of a finite field is cyclic.  

\begin{theorem}
$K$ field. $G$ finite subgroup of $K^*$ (multiplicative group of $K$), then $G$ cyclic.
\end{theorem}
\begin{proof}
The idea is to compare $G$ and the cyclic group of order $N$, $\mathbb{Z}/N\mathbb{Z}$, where $N=|G|$. Let $\psi(d)$ detote the number of elements of order $d$ in $G$. We need to prove that $\psi(N)\neq 0$. We know that $N=\sum \psi(d)$. Let $\phi(d)$ detote the number of elements of order $d$ in $\mathbb{Z}/N\mathbb{Z}$. 

$\mathbb{Z}/N\mathbb{Z}$ contains a single cyclic subgroup of order $d$, $d|N$ (the subgroup generated by $N/d$). $\phi(d)=$  number of generators of $\mathbb{Z}/N\mathbb{Z}=$ number of numbers between $1$ and $d - 1$, which are prime to $d$. So $\phi(n)\neq 0$. 

Claim: either $\psi(d)=0$ or $\psi(d)=\phi(d)$. This is sufficient since the $\sum \psi(s)=\sum\psi(d)$. Proof of the claim. If there is no element of order $d$ in $G$, then of course $\psi(d) =0$. If $\exists x\in G$, $x$ order $d$ $\implies$ $x$ is a root of the polynomial $x^d -1$. The roots of such a polynomial form a cyclic subgroup of $G$. So $G$, as well as $\mathbb{Z}/N\mathbb{Z}$ has a single subgroup of order d (which is cyclic) or no such subgroup at all. 

If $\phi(d)\neq 0$ $\implies$ there is such a subgroup. $\psi(d)=$ number of generators of that subgroup $=\phi(d)$. 

In particular $\psi(d)\leq \phi(d)$, but in fact there must be equality, because $\sum_{d|N}\psi(d)=\sum_{d|N}\phi(d)$ $\implies$ $\psi(d)=\phi(d)$.  In particular $\psi(n)\neq 0$. 
\end{proof}

\begin{corollary}
Let $K\supset \mathbb{F}_p$, $[K:\mathbb{F}_p]=n$ $\implies$ $\exists\alpha$ such that $K=\mathbb{F}_p(\alpha)$. In particular $\exists$ an irreducible polynomial of degree $n$ over $\mathbb{F}_p$.  
\end{corollary}

Well, you might object and say this has already been shown. We have seen that $F_(p^N)$ is a stem field of any irreducible polynomial of degree N, but this corollary is actually stronger because when we've been discussing those stem fields we did not say that such polynomials existed. 

\begin{proof}
It suffices to take $\alpha=$ generator of $K^*$.
\end{proof}

\begin{corollary}
The group of automorphisms of $\mathbb{F}_{p^n}$ over $\mathbb{F}_p$ is cyclic, generated by the Frobenius map, $F:x\mapsto x^p$.
\end{corollary}

\begin{proof}
$X^{p^n}=X$ $\forall x\in \mathbb{p^n}$ so $F^n=Id$. 

On the other hand the order of Frobenius is exactly $n$, $\text{ord}(F)=n$: If $m<n$, then $F^m\neq Id$ (because $x^{p^m} - x = 0$ has only $p^m$ roots and $p^m<p^n$). 

Finally $\mathbb{F}_{p^n}=\mathbb{F}_p(\alpha)$, where $\alpha$ is a root of irreducible polynomial of degree $n$, so it cannot have more than n automorphisms. This $\alpha$ goes to another root of $P$ under an automorphism of $\mathbb{F}_{p^n}$. So the cardinality of this group is at most $n$, $|\text{Aut}(\mathbb{F}_{p^n}/\mathbb{F}_p)|\leq n$ then it is $n$ and the group is cyclic generated by $F$.
\end{proof}

\section{Separable elements}

Our next topic is separable extensions. We would like to say that
a spliting polynomial spliting field of an irreducible
polynomial has many automorphisms. So 

$E$ spliting field of an irreducible polynomial "has many automorphisms":  if $\alpha,\beta$ roots of $P$, $E \supset K[\alpha]$, $E\supset K[\beta]$. $\exists$ isomorphism $\phi: K[alpha]\to K[beta]$ over $K$.  By extension theorem, I can extend it to an automorphism of $E$. 

There is one problem about this. Is it true, that an irreducible polynomial of degree $n$ has many ($n$) roots? The answer is yes in characteristic $0$. But not always if the characteristic of $K$ is a prime number $p$. An irreducible polynomial in characteristic $p$ can have multiple roots. This means that $\text{gcd}(P,P')\neq 1$. 

In characteristic 0 this cannot happen ($\text{deg }P<\text{deg }P'$ and $P'\neq 0$ when $P$ is nonconstant $\implies$ $P$ does not divide $P'$). 

In characteristic $p$, $P'$ can vanish. And then GCD can be equal to the polynomial $P$. How can $P'$ vanish? $P'$ vanishes exactly when $P$ is a polynomial in $x^p$. That is $P =\sum a_i x^i$, with $a_i\neq 0$ only if $i$ is divisible by $p$. Take $r = \text{max } h$ such that $P$ is a polynomial in $x^{p^h}$, that is, $a_i=0$ whenever $p^h$ does not divide $i$. $p^h$ does not divide i.

\begin{proposition}
$P(x)=Q(x^{p^r})$ and $Q'\neq 0$. In particular $(Q,Q')=1$ and $Q$ does not have multiple roots. In addition, all roots of $P$ have multiplicity $p^r$.
\end{proposition}

\begin{proof}
If $\lambda$ is a root of $P$, then $P = (x - \lambda) \cdot R$. Then $mu=\lambda^{p^r}$ is a root of $Q$. So $Q(y) =(y - lambda^{p^r})\cdot S$, where $\lambda$ is not a root of $S$. $\implies$ $P(x)=(x^{p^r}-\lambda^{p^r})S(x^{p^r})=(x-\lambda)^{p^r}S(x^{p^r})$ and $\lambda$ is not a root of $S(x^{p^r})$. $\implies$  The multiplicity of $\lambda$ is $p^r$.
\end{proof}

\begin{definition}
Let $P\in K[x]$ be irreducible. Then it is called \textbf{separable} if it is prime to its derivative, $(P,P')=1$ [so it does not have multiple roots]. The \textbf{separable degree} of $P$ is the degree of $Q$ is above, $d_{\text{sep}}(P)=\text{deg }Q$. The \textbf{degree of inseparability} of $P$ is the degree of $P$ over the degree of $Q$, $d_i(P)=\frac{\text{deg }P}{\text{deg }Q}(=p^r)$. $P$ is called \textbf{purely inseparable} if $\text{deg }P=d_i(P)$ (then $P=x^{p^r} - a$).
\end{definition}

\begin{definition}
$L$ algebraic extension of $K$. $\alpha \in L$ is called \textbf{separable} or \textbf{purely inseparable} over $K$ if $P_{\text{min}}(\alpha,K)$ has this property. 
\end{definition}

\begin{proposition}
If $\alpha$ is separable over $K$, then $|\text{Hom}_K(K(\alpha),\overbar{K})|=\text{deg }P_{\text{min}}(\alpha,K)$ (in general $|\text{Hom}_K(K(\alpha),\overbar{K})|=d_{\text{sep}}P_{\text{min}}(\alpha,K)$ ). 
\end{proposition}

\begin{proof}
The proof is obvious because the separable degree is just the number of distinct roots of this minimal polynomial. And we can send $\alpha$
into any of those roots.
\end{proof}

\section{Separable degree, separable extensions}

Let me generalize this property to the case of fields which are not necessarily given as $K(\alpha)$. 

$L$ finite extension of $K$. 
\begin{definition}
$[L:K]_{\text{sep}}=|\text{Hom}_K(L,\overbar{K})|$. (Again, if $L=K(\alpha)$ then this degree is just the number of distinct roots of its minimal polynomial).

The extension $L$ is separable over $K$ if $[L:K]_{\text{sep}}=[L:K]$.

$[L:K]_i=\frac{[L:K]}{[L:K]_{\text{sep}}}$.
\end{definition}

\begin{theorem}
\begin{enumerate}
\item $K\subset L\subset M$ $[M:K]_{\text{sep}} = [M:L]_{\text{sep}}\cdot [L:K]_{\text{sep}}$ and $M$ separable over $K$ $\iff$ $M$ separable over $L$ and $L$ separable over $K$.
\item TFAE:
\begin{enumerate}
\item $L$ is separable over $K$ 
\item $\forall \alpha \in L$, $\alpha$ is separable over $K$
\item $L$ is generated over $K$ by a finite number of separable elements, $L=K(\alpha_1,\alpha_2,\cdots\alpha_n)$
\item $L=K(\alpha_1,\alpha_2,\cdots\alpha_n)$ and $\alpha_i$ separable over $K(\alpha_1,\alpha_2,\cdots\alpha_{i-1})$
\end{enumerate}
\end{enumerate}
\end{theorem}

Remark: the same holds when we replace separability by pure inseparability.

\begin{proof}

\textit{Part one.} $\forall \phi: L \to \overbar{K}$ extends to $\widetilde{\phi}: M \to \overbar{K}$ (by the extension theorem). In fact, there are exactly $[M:L]_{\text{sep}}$ ways to do this. Since since given $\phi$, one considers $\overbar{K}$ as $\overbar{L}$. (An algebraic closure of
$K$ is also an algebraic closure of $L$ once an embedding of $L$ 
into $\overbar{K}$ is given.) Thus $[M:K]_{\text{sep}} = [M:L]_{\text{sep}}\cdot [L:K]_{\text{sep}}$.

Equivalence of separability: just the fact that the separable degree of an extension over K does not exceed the true degree, $[E:K]_{\text{sep}}\leq[E:K]$.

The last fact is proved by induction using the fact that this is true for $E=K(\alpha)$.

\textit{Part two.} (a) $\implies$ (b). The first part implies that any subextension $K(\alpha)$ of a separable extension $L$ is separable.

(b) $\implies$ (c). Obvious. If any element is separable then the generators are also separable.

(c) $\implies$ (d). This is clear because $P_{\text{min}}(a_i,K(\alpha_1,\cdots\alpha_{i-1}))$ divides $P_{\text{min}}(a_i,K)$ . If $P_{\text{min}}(a_i,K)$ ($\iff$ has distinct roots), so is its divisor $P_{\text{min}}(a_i,K(\alpha_1,\cdots\alpha_{i-1}))$.

(d) $\implies$ (a). This can be proved by induction as above.
\end{proof}

One might ask, is the notion of separability defined for extensions which are not necessarily finite? Yes, in this case it is best to define a separable extension as such extension that all its elements are separable. 

In particular, if $L$ over $K$ is not necessarily finite and algebraic extension, we can define $L$ separable, the separable closure of $K$ in $L$, as $L^{\text{sep}}=\left\{ x|x\text{ separable over }K\right\}$. (A separable element is by definition algebraic, it has minimal polynomial). So the preceding theorem implies that this $L^{\text{sep}}$ is a subfield, a subextension called \textbf{separable closure} of $K$ in $L$. $L$ is purely inseparable over $L^{\text{sep}}$. 

Remark: if the $\text{char }K=0$, then any extension is separable. And if  $\text{char }K=p$, then a purely inseparable extension has degree $p^r$, and always this degree of inseparability is $[L:K]_i=p^r$.

\section{Perfect fields}

Let K be a field with $\text{char }K=p>0$, $p$ prime number. 

\begin{definition}
$K$ is \textbf{perfect} if the Frobenius automorphism, $F:K\to K$, $x \mapsto x^p$, is surjective.
\end{definition}

\begin{example}
A finite field is always perfect. Since an injective self-map of a finite set is surjective.
\end{example}

\begin{example}
An algebraically closed field is perfect. Since $x^p-a$ has a root $\alpha$ for any $a$, so in particular, $a=F(\alpha)$.
\end{example}

\begin{example}
Take $K=\mathbb{F}_p(x)$ (this is a variable $x$) the field of rational functions in one variable $f(x)/g(x)$, where $f,g\in\mathbb{F}_p[x]$, over $\mathbb{F}_p$. $\text{Im }F= \mathbb{F}_p(x^p)\neq \mathbb{F}_p(x)$. $\implies$ $K$ is not perfect.
\end{example}

\begin{theorem}
$K$ is perfect $iff$ all irreducible polynomials over $K$ are separable $\iff$ all algebraic extensions of $K$ are separable. (Separability only makes sense for algebraic extensions.)
\end{theorem}

\begin{proof}
$\Rightarrow$: Suppose $K$ perfect. Let me take an irreducible polynomial $P\in K[x]$, suppose that $P$ is a polynomial in some power of $x$, $P(x)=Q(x^{p^r})=\sum a_i (x^{p^r})^i$. Since $K$ perfect, we can extract p-th roots of $a_i$'s and we can do it repeatedly. So $\exists b_i \in K$, such that $b^{p^r} = a_i$. So $P=(\sum b_i x^i)^{p^r}$ is not irreducible unless $r = 0$. If it's irreducible then it's separable. 

$\Leftarrow$: If $K$ is not perfect, then $\exists a\notin \text{Im }F$. Then $x^{p^r} - a$ is irreducible. In fact, all roots in $\overbar{K}$ are the same $x$ with $x^{p^r} = a$. And $x^{p^{r-1}}\notin K$. We have already seen that in this case the degree of $K(x)$ over $K$ is exactly $p^r$, so this polynomial is irreducible. 
\end{proof}
